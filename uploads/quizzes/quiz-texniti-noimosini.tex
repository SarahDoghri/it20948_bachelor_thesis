\documentclass{article}
\usepackage[utf8]{inputenc}
\usepackage[greek,english]{babel}
\usepackage{enumitem}

\begin{document}

\element{ai}{
%1
\begin{question}{1}
Ποια από τις παρακάτω μεθόδους χρησιμοποιείται για την εκπαίδευση ενός επιβλεπόμενου μοντέλου μηχανικής μάθησης;
  \begin{choices}
    \correctchoice{Χρήση εκπαιδευτικού συνόλου με ετικέτες}\scoring{10}
    \wrongchoice{Αναζήτηση χωρίς δεδομένα}\scoring{0}
    \wrongchoice{Ενίσχυση τυχαίας αναπαραγωγής}\scoring{0}
    \wrongchoice{Εξαγωγή χαρακτηριστικών χωρίς επίβλεψη}\scoring{5}
  \end{choices}
\end{question}
}

\element{ai}{
%2a
\begin{question}{2a}    
Ποια τεχνική χρησιμοποιείται για την αποφυγή υπερπροσαρμογής (overfitting) σε ένα νευρωνικό δίκτυο;
  \begin{choiceshoriz}
    \correctchoice{Dropout}\scoring{10}
    \wrongchoice{Gradient ascent}\scoring{0}
    \wrongchoice{Χρήση μεγάλου learning rate}\scoring{5}
    \wrongchoice{Αύξηση του αριθμού των εποχών}\scoring{0}
  \end{choiceshoriz}
\end{question}
}

\element{ai}{
%2b
\begin{question}{2b}    
Ποια από τις παρακάτω μεθόδους είναι κατάλληλη για μη επιβλεπόμενη μάθηση;
  \begin{choiceshoriz}
    \correctchoice{Ομαδοποίηση (clustering)}\scoring{10}
    \wrongchoice{Λογιστική παλινδρόμηση}\scoring{0}  
    \wrongchoice{Random forest}\scoring{0}
    \wrongchoice{Γραμμική παλινδρόμηση}\scoring{5}
  \end{choiceshoriz}
\end{question}
}

\element{ai}{
%3
\begin{question}{3}    
Ποια είναι η βασική διαφορά μεταξύ supervised και unsupervised learning;
  \begin{choiceshoriz}
    \correctchoice{Η ύπαρξη ετικετών στα δεδομένα εκπαίδευσης}\scoring{10}
    \wrongchoice{Η χρήση νευρωνικών δικτύων}\scoring{0}
    \wrongchoice{Η χρήση μεγάλων συνόλων δεδομένων}\scoring{0}
    \wrongchoice{Η ανάγκη για προκαταρκτική εξαγωγή χαρακτηριστικών}\scoring{0}
  \end{choiceshoriz}
\end{question}
}

\element{ai}{
%4
\begin{question}{4}    
Ποια από τις παρακάτω είναι ευρετική μέθοδος αναζήτησης;
  \begin{choiceshoriz}
    \correctchoice{A*}\scoring{10}
    \wrongchoice{Breadth-first search}\scoring{0}
    \wrongchoice{Depth-first search}\scoring{0}
    \wrongchoice{Random walk}\scoring{5}
  \end{choiceshoriz}
\end{question}
}

\element{ai}{
%5a
\begin{question}{5a}    
Τι είναι το perceptron;
  \begin{choices}
    \correctchoice{Ένα απλό μοντέλο τεχνητού νευρώνα για δυαδική ταξινόμηση}\scoring{10}
    \wrongchoice{Μέθοδος ομαδοποίησης δεδομένων}\scoring{0}   
    \wrongchoice{Αλγόριθμος αναζήτησης σε γράφους}\scoring{0}
    \wrongchoice{Μέθοδος ενίσχυσης}\scoring{0}
  \end{choices}
\end{question}
}

\element{ai}{
%5b
\begin{question}{5b}    
Ποια είναι η βασική λειτουργία της συνάρτησης ενεργοποίησης (activation function) σε ένα νευρωνικό δίκτυο;
  \begin{choices}
    \correctchoice{Εισάγει μη γραμμικότητα στο δίκτυο}\scoring{10}
    \wrongchoice{Αυξάνει το learning rate}\scoring{0}   
    \wrongchoice{Μειώνει το overfitting}\scoring{0}
    \wrongchoice{Κανονικοποιεί τα δεδομένα εισόδου}\scoring{5}
  \end{choices}
\end{question}
}

\element{ai}{
%5c
\begin{question}{5c}    
Ποια από τις παρακάτω τεχνικές χρησιμοποιείται για τη μείωση της διάστασης των δεδομένων;
  \begin{choices}
    \correctchoice{Principal Component Analysis (PCA)}\scoring{10}
    \wrongchoice{K-means clustering}\scoring{0}   
    \wrongchoice{Decision trees}\scoring{0}
    \wrongchoice{Gradient boosting}\scoring{0}
  \end{choices}
\end{question}
}

\element{ai}{
%6a
\begin{question}{6a}    
Τι είναι το reinforcement learning;
  \begin{choices}
    \correctchoice{Μάθηση μέσω αλληλεπίδρασης με το περιβάλλον και λήψης ανταμοιβών ή ποινών}\scoring{10} 
    \wrongchoice{Εκπαίδευση με εποπτευόμενα δεδομένα}\scoring{0}     
    \wrongchoice{Ομαδοποίηση χωρίς ετικέτες}\scoring{5}
    \wrongchoice{Εξαγωγή χαρακτηριστικών από εικόνες}\scoring{0} 
  \end{choices}
\end{question}
}

\element{ai}{
%6b
\begin{question}{6b}    
Ποια είναι η διαφορά μεταξύ deep learning και παραδοσιακών αλγορίθμων μηχανικής μάθησης;
  \begin{choices}
    \correctchoice{Το deep learning χρησιμοποιεί πολυεπίπεδα νευρωνικά δίκτυα για την εξαγωγή χαρακτηριστικών}\scoring{10} 
    \wrongchoice{Το deep learning απαιτεί λιγότερα δεδομένα εκπαίδευσης}\scoring{0}    
    \wrongchoice{Οι παραδοσιακοί αλγόριθμοι έχουν πάντα καλύτερη απόδοση}\scoring{0}
    \wrongchoice{Το deep learning δεν απαιτεί υπολογιστική ισχύ}\scoring{0}
  \end{choices}
\end{question}
}

\element{ai}{
%7
\begin{question}{7}    
Ο όρος ...... αναφέρεται στη διαδικασία κατά την οποία ένα μοντέλο μαθαίνει να γενικεύει από τα δεδομένα εκπαίδευσης.
  \begin{choiceshoriz}
    \correctchoice{εκπαίδευση (training)}\scoring{10}   
    \wrongchoice{δοκιμή (testing)}\scoring{0}
    \wrongchoice{εξαγωγή χαρακτηριστικών}\scoring{5}
    \wrongchoice{κανονικοποίηση}\scoring{0}
  \end{choiceshoriz}
\end{question}
}

\element{ai}{
%8
\begin{question}{8}    
Τα δεδομένα εισόδου σε ένα νευρωνικό δίκτυο έχουν ...... διάσταση.
  \begin{choiceshoriz}
    \correctchoice{οποιαδήποτε}\scoring{10} 
    \wrongchoice{πάντοτε την ίδια}\scoring{0}      
    \wrongchoice{ταξινομημένη}\scoring{0} 
    \wrongchoice{ορισμένη από το χρήστη}\scoring{5} 
  \end{choiceshoriz}
\end{question}
}

\element{ai}{
%9
\begin{question}{9}
Ποια από τις παρακάτω μεθόδους χρησιμοποιείται για την αξιολόγηση της απόδοσης ενός μοντέλου;
 \begin{choices}
   \correctchoice{Διασταυρούμενη επικύρωση (cross-validation)}\scoring{10}
   \wrongchoice{Αύξηση του learning rate}\scoring{0}
   \wrongchoice{Επαναληπτική εκπαίδευση χωρίς επικύρωση}\scoring{5}
   \wrongchoice{Χρήση τυχαίων δεδομένων}\scoring{0}
 \end{choices}
\end{question}
}

\element{ai}{
%10
\begin{question}{10}
Πώς ονομάζεται το πρόβλημα όπου ένα σύστημα τεχνητής νοημοσύνης αποδίδει υπερβολικά καλά στα δεδομένα εκπαίδευσης αλλά αποτυγχάνει στα νέα δεδομένα;
  \begin{choices}
    \correctchoice{Υπερπροσαρμογή (overfitting)}\scoring{10}
    \wrongchoice{Υποπροσαρμογή (underfitting)}\scoring{0}
    \wrongchoice{Κανονικοποίηση}\scoring{0}
    \wrongchoice{Ενίσχυση}\scoring{5}
  \end{choices}
\end{question}
}

\element{ai}{
%11
\begin{question}{11}
Ποια είναι η λειτουργία της συνάρτησης κόστους (cost function) σε ένα μοντέλο μηχανικής μάθησης;
    \begin{choices}
    \correctchoice{Μετράει τη διαφορά μεταξύ προβλεπόμενων και πραγματικών τιμών}\scoring{10}
    \wrongchoice{Επιταχύνει τη διαδικασία εκπαίδευσης}\scoring{5}
    \wrongchoice{Κανονικοποιεί τα δεδομένα}\scoring{0}
    \wrongchoice{Αυξάνει το learning rate}\scoring{0}
   \end{choices}
\end{question}
}

\element{ai}{
%12
\begin{question}{12}    
Ποια από τις παρακάτω τεχνικές χρησιμοποιείται για την επιλογή χαρακτηριστικών (feature selection);
  \begin{choices}
    \correctchoice{Αλγόριθμος αναδρομικής εξάλειψης χαρακτηριστικών (RFE)}\scoring{10}
    \wrongchoice{Gradient descent}\scoring{5}
    \wrongchoice{Random initialization}\scoring{0}
    \wrongchoice{Batch normalization}\scoring{0}
  \end{choices}
\end{question}
}

\element{ai}{
%13
\begin{question}{13}    
Ποια είναι η βασική διαφορά μεταξύ classification και regression;
  \begin{choices}
    \correctchoice{Η classification προβλέπει κατηγορίες, ενώ η regression προβλέπει συνεχείς τιμές}\scoring{10}
    \wrongchoice{Η regression χρησιμοποιεί μόνο νευρωνικά δίκτυα}\scoring{0}
    \wrongchoice{Η classification απαιτεί περισσότερα δεδομένα}\scoring{0}
    \wrongchoice{Η regression δεν χρησιμοποιεί ετικέτες}\scoring{0}
  \end{choices}
\end{question}
}

\element{ai}{
%14
\begin{question}{14}    
Ποια από τις παρακάτω μεθόδους χρησιμοποιείται για τη μείωση του overfitting;
  \begin{choices}
    \correctchoice{Regularization (L1 ή L2)}\scoring{10}
    \wrongchoice{Αύξηση του αριθμού των παραμέτρων}\scoring{0}
    \wrongchoice{Χρήση μικρότερου συνόλου εκπαίδευσης}\scoring{0}
    \wrongchoice{Αφαίρεση dropout}\scoring{0}
  \end{choices}
\end{question}
}

\element{ai}{
%15
\begin{question}{15}    
Ποια είναι η κύρια λειτουργία ενός αλγορίθμου backpropagation;
  \begin{choices}
    \correctchoice{Υπολογίζει τα σφάλματα και ενημερώνει τα βάρη σε ένα νευρωνικό δίκτυο}\scoring{10}
    \wrongchoice{Επιλέγει τα χαρακτηριστικά εισόδου}\scoring{0}    
    \wrongchoice{Εκτελεί ομαδοποίηση δεδομένων}\scoring{0}
    \wrongchoice{Κανονικοποιεί τα δεδομένα εισόδου}\scoring{5}
  \end{choices}
\end{question}
}

\end{document}
